\chapter*{Conclusions}
\addcontentsline{toc}{chapter}{Conclusions}
\chaptermark{Conclusions}

Image composition is among the most common types of image manipulation and consists of using parts of two or more images to create a new fake one. In this context, this work has presented a method, composed by two different modules, that relies on illumination inconsistencies for detecting this image forgeries.

In the two different approaches, divided into two modules, proposed in Chapter 2, we analyzed illuminant maps entailing the interaction between the light source and the objects contained in a scene. 

The first module, aimed at detecting forgeries involving people, is based on the assumption that similar materials (\emph{e.g.} the human skin) illuminated by a common light source have similar properties in such maps. Features based on color of illuminant maps are now used to describe human face regions and a set of kNN models are trained and used for classification using a fusion technique. However, although image composition involving people is one of the most usual ways of modifying images, other elements can also be inserted into them. To address this issue, we combine the first module with another kind of approach, still based on the illuminant maps analysis, but completely image content independent. This second approach is based on considering a set of different illuminant color estimators (derived from the GGE framework presented in Chapter 1) and a simple image segmentation. Images are divided into two types of stripes, vertical and horizontal and then each stripe is encoded into a 5-dimensional feature vector composed by all the differences between the evaluated illuminant color for the analyzed band and a reference one. Band classification has been made using SVM classifiers.

This face module achieved the most promising results, but it needs some a priori knowledge about the content of the image which will be analyzed and, above all, it operates only in presence of human faces. The second approach used has instead brought encouraging results. The method is not reliable in the detection of forgeries due to the large number of false positives obtained in classification, but it can be a starting point for a retrospectively analysis of the image by an expert.

Future developments of this work may include the extension of the first module considering additional and different parts of body (\emph{e.g.}, all skin spots of the human body visible in an image). Given that our method compares skin material, it is feasible to use additional body parts, such as arms and legs, to increase the detection and confidence of the method.

Although promising as forensic evidence, methods that operate on illuminant colors are inherently prone to estimation errors. Thus, we expect that further improvements can be achieved when more advanced illuminant color estimators become available.