\chapter*{Conclusions}
\addcontentsline{toc}{chapter}{Conclusions}
\chaptermark{Conclusions}

Image composition is among the most common types of image manipulation and consists of using parts of two or more images to create a new fake one. In this context, this work has presented a method, composed by two different modules, that relies on illumination inconsistencies for detecting this image forgeries.

In the two different approaches, divided into two modules, proposed in Chapter 2, we analyzed illuminant maps entailing the interaction between the light source and the objects contained in a scene. The first module, aimed at detecting forgeries involving people, is based on the assumption that similar materials (\emph{e.g.} the human skin) illuminated by a common light source have similar properties in such maps.

As for research directions and future work, we suggest different contributions for each one of our proposed approaches. Future developments of this work may include the extension of the first module considering additional and different parts of body (\emph{e.g.}, all skin spots of the human body visible in an image). Given that our method compares skin material, it is feasible to use additional body parts, such as arms and legs, to increase the detection and confidence of the method.

Although promising as forensic evidence, methods that operate on illuminant color are inherently prone to estimation errors. Thus, we expect that further improvements can be achieved when more advanced illuminant color estimators become available.