\chapter{Related work}

\section{Image splicing}



Images and videos have become the main information carriers in the digital era and used to store real world events. The significant possible of visual media and the no difficulty in their acquisition, division and storage is such that they are more and more exploited to convey information. 
But digital images are easy to manipulate because of the availability of the powerful editing software and sophisticated digital cameras. 

Image processing experts can easily access and modify image content and therefore its meaning without leaving visually detectable traces. Moreover, with the spread of low-cost user friendly editing tools the art of tampering and counterfeiting visual content is no more restricted to experts. As a result, the manipulation of images for malicious purposes is now more common than ever. 

At the start, the manipulation is just improve the image's performance, but then many people started to change the image's content, even to gain their ends by these illegal and immorality methods. Based on the above reasons, it is important to develop a credible method to detect whether a digital image is tempered, so-called digital image forgery.

Digital imaging resulted into many real life benefits, but at same side it's vulnerable to many threats of crimes. To check whether image is real or forged? , it is very difficult task. In fact, the security concern of digital content has arisen a long time ago and different techniques for validating the integrity of digital images have been developed. These techniques can be divided into two major groups: intrusive and non-intrusive. In intrusive (active) techniques, some sort of signature (watermark, extrinsic fingerprint) is embedded into a digital image, and authenticity is established by verifying if the true signature matches the retrieved signature from the test image [3] [4] [5]. This approach is limited due to the inability of many digital cameras and video recorders available in the market to embed extrinsic fingerprints [6]. Further the drawbacks of intrusive methods used as motivation for non-intrusive method [6] in order to validate the authenticity of digital images. These techniques exploit different kinds of intrinsic fingerprints such as sensor noise of the capturing device or image specific detectable changes for detecting forgery. There are many challenges in blind techniques, for instance, reducing false positive rates (i.e., an authentic image being detected as a forged image), making the system fully automated, localizing the forgery, detecting forgery of any type of image format (compressed or uncompressed), increasing the robustness and reliability, etc.
Image splicing is to create a new image from two or more images, and it is far and wide used for image forgery. Image splicing detection is a main difficulty in image forensics. However there are very hardly any solutions to this problem .An given below example of a digital forgery is shown in Figure 1. As the newspaper cutout shows, three different photographs were used in creating the composite image: Image of the White House, Bill Clinton, and Saddam Hussein. The White House was rescaled and blurred to create an illusion of an out-of-focus background. Then, Bill Clinton and Saddam were cut off from two different images and pasted on the White House image. Care was taken to bring in the speaker stands with microphones while preserving the correct shadows and lighting. Fig. 1 is, in fact, an example of a very realistic looking forgery.

\section{Methods based on light inconsistencies}